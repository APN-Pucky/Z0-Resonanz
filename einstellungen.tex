%%% Einstellungen zur richtigen Benutzung von wwustyle.sty
\usefonttheme{professionalfonts}

% Einstellungen der Schriftart (Meta Office Pro) für Text und Mathematik
% math**=sym angeben, damit auch diese Befehle die Schriftart Meta verwenden
\usepackage[mathrm=sym, mathit=sym, mathsf=sym, mathbf=sym]{unicode-math}
\setmainfont{MetaOffcPro}[Path=fonts/,
Extension=.ttf,
UprightFont=*-Norm,
UprightFeatures={
	SmallCapsFont={MetaScOffcPro-Norm}
},
ItalicFont=*-Norm,
ItalicFeatures={
	SmallCapsFont={MetaScOffcPro-NormIta}
},
BoldFont=*-Bold,
BoldFeatures={
	SmallCapsFont={MetaScOffcPro-Bold}
},
BoldItalicFont=*-Bold,
BoldItalicFeatures={FakeSlant},
]
\setsansfont{MetaOffcPro}[Path=fonts/,
Extension=.ttf,
UprightFont=*-Norm,
UprightFeatures={
	SmallCapsFont={MetaScOffcPro-Norm}
},
ItalicFont=*-Norm,
ItalicFeatures={
	SmallCapsFont={MetaScOffcPro-NormIta}
},
BoldFont=*-Bold,
BoldFeatures={
	SmallCapsFont={MetaScOffcPro-Bold}
},
BoldItalicFont=*-Bold,
BoldItalicFeatures={FakeSlant},
]
\setmathfont{Latin Modern Math}
% Meta Office Pro für die Bereiche nutzen, für die Glyphen existieren
\setmathfont{MetaOffcPro-Norm.ttf}[Path=fonts/,
range=up/{greek,Greek,latin,Latin,num}]
\setmathfont{MetaOffcPro-NormIta.ttf}[Path=fonts/,
range=it/{greek,Greek,latin,Latin,num}]
\setmathfont{MetaOffcPro-Bold.ttf}[Path=fonts/,
range=bfup/{greek,Greek,latin,Latin,num}]
\setmathfont{MetaOffcPro-Bold.ttf}[Path=fonts/, UprightFeatures={FakeSlant},
range=bfit/{greek,Greek,latin,Latin,num}]
% Symbole (leider enthält Meta Office Pro nicht das Symbol ∓)

%\setmathfont{MetaOffcPro-Norm.ttf}[Path=fonts/]


% Spracheinstellung
\usepackage{polyglossia}
\setmainlanguage{german}

% Offizielles WWU-LaTeX-Paket für Präsentation (leicht modifiziert)
% Mögliche Optionen:
% - english: Verwendet englischen Claim („living.knowledge“)
% - Verschiedene Farbvarianten:
%   pantoneblack7, pantone312, pantone7462, pantone3135, pantone315, pantone369,
%   pantone390, pantone396, pantone3282, pantoneprozessyellow
% - Verschiedene Titel-Motive:
%   - belltower (Standard-Wert): Glockenturm des Schlosses
%   - wedge: Textkeil
%   - prinz: WWU-Schriftzug auf dem Prinzipalmarkt (Foto)
% - inverse: Inverses Titelbild (Weiß auf farbigem Hintergrund statt Farbe auf
%            weißem Hintergrund)
% - wide: Seitenverhältnis 16:10 verwenden (statt Standardwert 4:3)
\usepackage[pantone312]{wwustyle-mod}
% Typographische Verbesserungen (Verbot mancher Ligaturen)
\usepackage{selnolig}
% Typographische Verbesserungen (Mikrotypographie)
\usepackage{microtype}

% Daten/Zeiten formatieren
\usepackage[useregional]{datetime2}
% Formatierung von Telefonnummern
\usepackage{phonenumbers}
% „Schöne“ Brüche im Fließtext mit \sfrac
\usepackage{xfrac}
% Ermöglicht die Nutzung von „\SI{Zahl}{Einheit}“
\usepackage{siunitx}
% Automatisches Umwandeln von Anführungszeichen
\usepackage{csquotes}

% Farben ermöglichen
\usepackage{xcolor}
% Paket für Bilder-Einbindung (EPS, PNG, JPG, PDF)
\usepackage{graphicx}
% .tex-Dateien mit \includegraphics einbinden
\usepackage{gincltex}
% Bessere Verarbeitung von Dateinamen für \includegraphics etc.
\usepackage{grffile}


% latex
\renewcommand{\arraystretch}{1.3}
% graphicx
% Standardmäßig „keepaspectratio“ verwenden
% s. https://tex.stackexchange.com/a/91619/51235
\setkeys{Gin}{keepaspectratio}
% hyperref
\hypersetup{unicode}
% siunitx
\sisetup{
	locale=DE,
	binary-units,
	quotient-mode=fraction,
	per-mode=fraction,
	fraction-function=\sfrac,
	detect-weight
}
% csquotes
\MakeOuterQuote{"}

%\institutelogo{\raisebox{-5.75mm}{\includegraphics[width=3.8cm]{fsphys-logo.pdf}}}
%\institutelogosmall{\raisebox{-2.5mm}[0pt][0pt]{\includegraphics[width=2.6cm]{fsphys-logo.pdf}}}

%%%%%%%%%%%%%%%%%%%%%%%%%%%%%%%%%%%%%%%

% Zusätzliche Einstellungen/Befehle
\let\strong\textbf
\newcommand{\email}[1]{\href{mailto:#1}{\texttt{#1}}}
\newfontfamily\DejaSans{DejaVu Sans}

\usepackage{nameref}
\makeatletter
\newcommand*{\currentname}{\@currentlabelname}
\makeatother

\newenvironment{iframe}{
	\begin{frame}
	\frametitle{\subsecname}	
	%\framesubtitle{\subsecname}
}{\end{frame}}

\newenvironment{nframe}{
	\begin{frame}[noframenumbering]
	\frametitle{\subsecname}	
	%\framesubtitle{\subsecname}
}{\end{frame}}

\newcommand{\foo}{\makebox[0pt]{\textbullet}\hskip-0.5pt\vrule width 1pt\hspace{\labelsep}}


\usepackage{chronology}


\usepackage{tikz}

\usetikzlibrary{arrows, calc, decorations.markings, positioning}


\makeatletter
\newenvironment{timeline}[6]{%
	% #1 is startyear
	% #2 is tlendyear
	% #3 is yearcolumnwidth
	% #4 is rulecolumnwidth
	% #5 is entrycolumnwidth
	% #6 is timelineheight
	
	\newcommand{\startyear}{#1}
	\newcommand{\tlendyear}{#2}
	
	\newcommand{\yearcolumnwidth}{#3}
	\newcommand{\rulecolumnwidth}{#4}
	\newcommand{\entrycolumnwidth}{#5}
	\newcommand{\timelineheight}{#6}
	
	\newcommand{\templength}{}
	
	\newcommand{\entrycounter}{0}
	
	% https://tex.stackexchange.com/questions/85528/checking-whether-or-not-a-node-has-been-previously-defined
	% https://tex.stackexchange.com/questions/37709/how-can-i-know-if-a-node-is-already-defined
	\long\def\ifnodedefined##1##2##3{%
		\@ifundefined{pgf@sh@ns@##1}{##3}{##2}%
	}
	
	\newcommand{\ifnodeundefined}[2]{%
		\ifnodedefined{##1}{}{##2}
	}
	
	\newcommand{\drawtimeline}{%
		\draw[timelinerule] (\yearcolumnwidth+5pt, 0pt) -- (\yearcolumnwidth+5pt, -\timelineheight);
		\draw (\yearcolumnwidth+0pt, -10pt) -- (\yearcolumnwidth+10pt, -10pt);
		\draw (\yearcolumnwidth+0pt, -\timelineheight+15pt) -- (\yearcolumnwidth+10pt, -\timelineheight+15pt);
		
		\pgfmathsetlengthmacro{\templength}{neg(add(multiply(subtract(\startyear, \startyear), divide(subtract(\timelineheight, 25), subtract(\tlendyear, \startyear))), 10))}
		\node[year] (year-\startyear) at (\yearcolumnwidth, \templength) {\startyear};
		
		\pgfmathsetlengthmacro{\templength}{neg(add(multiply(subtract(\tlendyear, \startyear), divide(subtract(\timelineheight, 25), subtract(\tlendyear, \startyear))), 10))}
		\node[year] (year-\tlendyear) at (\yearcolumnwidth, \templength) {\tlendyear};
	}
	
	\newcommand{\entry}[2]{%
		% #1 is the year
		% #2 is the entry text
		
		\pgfmathtruncatemacro{\lastentrycount}{\entrycounter}
		\pgfmathtruncatemacro{\entrycounter}{\entrycounter + 1}
		
		\ifdim \lastentrycount pt > 0 pt%
		\node[entry] (entry-\entrycounter) [below of=entry-\lastentrycount] {##2};
		\else%
		\pgfmathsetlengthmacro{\templength}{neg(add(multiply(subtract(\startyear, \startyear), divide(subtract(\timelineheight, 25), subtract(\tlendyear, \startyear))), 10))}
		\node[entry] (entry-\entrycounter) at (\yearcolumnwidth+\rulecolumnwidth+10pt, \templength) {##2};
		\fi
		
		%\ifnodeundefined{year-##1}
		
		\draw ($(year-##1.east)+(2.5pt, 0pt)$) -- ($(year-##1.east)+(7.5pt, 0pt)$) -- ($(entry-\entrycounter.west)-(5pt,0)$) -- (entry-\entrycounter.west);
	}
	
	\newcommand{\plainentry}[2]{% plainentry won't print date in the timeline
		% #1 is the year
		% #2 is the entry text
		
		\pgfmathtruncatemacro{\lastentrycount}{\entrycounter}
		\pgfmathtruncatemacro{\entrycounter}{\entrycounter + 1}
		
		\ifdim \lastentrycount pt > 0 pt%
		\node[entry] (entry-\entrycounter) [below of=entry-\lastentrycount] {##2};
		\else%
		\pgfmathsetlengthmacro{\templength}{neg(add(multiply(subtract(\startyear, \startyear), divide(subtract(\timelineheight, 25), subtract(\tlendyear, \startyear))), 10))}
		\node[entry] (entry-\entrycounter) at (\yearcolumnwidth+\rulecolumnwidth+10pt, \templength) {##2};
		\fi
		
		\ifnodeundefined{invisible-year-##1}{%
			\pgfmathsetlengthmacro{\templength}{neg(add(multiply(subtract(##1, \startyear), divide(subtract(\timelineheight, 25), subtract(\tlendyear, \startyear))), 10))}
			\draw (\yearcolumnwidth+2.5pt, \templength) -- (\yearcolumnwidth+7.5pt, \templength);
			\node[year] (invisible-year-##1) at (\yearcolumnwidth, \templength) {};
		}
		
		\draw ($(invisible-year-##1.east)+(2.5pt, 0pt)$) -- ($(invisible-year-##1.east)+(7.5pt, 0pt)$) -- ($(entry-\entrycounter.west)-(5pt,0)$) -- (entry-\entrycounter.west);
	}
	
	\begin{tikzpicture}
	\tikzstyle{entry} = [%
	align=left,%
	text width=\entrycolumnwidth,%
	node distance=10mm,%
	anchor=west]
	\tikzstyle{year} = [anchor=east]
	\tikzstyle{timelinerule} = [%
	draw,%
	decoration={markings, mark=at position 1 with {\arrow[scale=1.5]{latex'}}},%
	postaction={decorate},%
	shorten >=0.4pt]
	
	\drawtimeline
}
{
	\end{tikzpicture}
	\let\startyear\@undefined
	\let\tlendyear\@undefined
	\let\yearcolumnwidth\@undefined
	\let\rulecolumnwidth\@undefined
	\let\entrycolumnwidth\@undefined
	\let\timelineheight\@undefined
	\let\entrycounter\@undefined
	\let\ifnodedefined\@undefined
	\let\ifnodeundefined\@undefined
	\let\drawtimeline\@undefined
	\let\entry\@undefined
}
\makeatother

\usepackage{svg}

\usepackage{biblatex}
\usepackage{caption}
\captionsetup{font=scriptsize,labelfont=scriptsize}

\usepackage{amsmath}
\usepackage{amsfonts}
\usepackage{amssymb}
    \newcommand{\bra}[1]{\ensuremath{\left\langle#1\right|}}
\newcommand{\ket}[1]{\ensuremath{\left|#1\right\rangle}}
\newcommand{\bracket}[2]{\ensuremath{\left\langle #1 \middle| #2 \right\rangle}}
\newcommand{\matrixel}[3]{\ensuremath{\left\langle #1 \middle| #2 \middle| #3 \right\rangle}}

\usepackage[absolute,overlay]{textpos}

\usepackage{pgfpages}
%\setbeameroption{show only notes}
\setbeameroption{show notes on second screen}
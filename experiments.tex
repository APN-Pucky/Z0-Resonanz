\subsection{Erzeugung}
\begin{iframe}
	\begin{columns}
		\begin{column}{0.48\textwidth}
				\begin{figure}
				\feynmandiagram [large,horizontal=a to b] {
					i1 -- [fermion,edge label=\(e^-\)] a -- [fermion,edge label=\(e^+\)] i2,
					a -- [boson,edge label=\(\gamma\)] b,
					f1 -- [fermion,edge label'=\(\overline f\)] b -- [fermion,edge label'=\(f\)] f2,
				};
				\caption*{$e^+e^$-Vernichtung über $\gamma$ \cite{perkins}}
			\end{figure}
		\end{column}
		\begin{column}{0.48\textwidth}
			\begin{figure}
				\feynmandiagram [large,horizontal=a to b] {
					i1 -- [fermion,edge label=\(e^-\)] a -- [fermion,edge label=\(e^+\)] i2,
					a -- [boson,edge label=\(Z^0\)] b,
					f1 -- [fermion,edge label'=\(\overline f\)] b -- [fermion,edge label'=\(f\)] f2,
				};
			\caption*{$e^+e^$-Vernichtung über $Z^0$ \cite{perkins}}
			\end{figure}
		\end{column}
	\end{columns}
\end{iframe}
\note[itemize] {
	\item feynman diagram %TODO
	\item bei passender Energie dominiert $Z^0$ %TODO
}
\begin{iframe}
	\begin{itemize}
		\item Schwerpunktsenergie $\sqrt{s} = 2E_e \geq M_\text{Z}c^2 \approx \SI{91.6}{GeV}$ 
		\pause
		\item $e^+ + e^- \rightarrow W^+ + W^-$ benötigt $\sqrt{s} \geq 2M_\text{W}c^2 \approx \SI{160.8}{GeV}$
	\end{itemize}
\note[item]{ 1989 am Stanford Linear Collider}
\note[item]{ 1996 am LEP}
\end{iframe}


\subsection{LEP am CERN}